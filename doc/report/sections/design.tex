This section describes shortly how the Cristin API was implemented. It then describes the process of crawling Cristins rdbm. Finally, it describes how the extracted data was processed to create relations and nodes in a graph database.

% Skakke ha kake a kai
\subsection*{Cristin API}

% Skakke ha kake a kai
\subsection*{Crawling Cristin}

% Eznis
\subsection*{Processing extracted data}
Data extracted from Cristin is structured in json format which makes it easy to work with. Json objects are a key-value store which makes it similar to nodes in a graph database. Nodes are the entities in a graph database. They can hold any number of attributes (key-value-pairs) called properties. Nodes can be tagged with labels representing their different roles in your domain.\cite{neo4j} The difference between a json object and a node is that a json value can take form of a \textit{dictionary} and a property value can't. To make a json object compatible as a node be sure that every value isn't a dictionary and attach a label to it.





% Eznis
\subsection*{Inserting data}
