The crawler and system for inserting data into the graph database is implemented using python.
Neo4j 3.3 is the graph database used for storing and querying the data.


There are a number of ways to evaluate research connectivity, but for this paper, the following data is examined.
The total number of publications authored at the five top publishing institutions in Norway, and whether or not they are subject to cooperation between researchers from different institutions, such as UiO and NTNU.
This will be a measure of the connectivity between institutions in Norway, and it remains to measure the connectivity within institutions.
Connectivity within institutions can be measured in much the same way, but rather than determining if results have been subject to cooperation between researchers from different institutions, we want to determine if they are produced by persons within the same institutions but from different units of that institution, such as a different faculty, or a different department.

The sample size is 1.17 million nodes, comprising researchers, results, institutions and units. This is not all data available, but we belive the sample size makes this data representative of the population as a whole.

\begin{figure}[h]
	\centering
	\begin{tabular}{| l || c | c |}
		\hline
		Institution	& Only same institution	& Cross institution	\\ \hline
		UiO		& 0.329			& 0.671			\\
		NTNU		& 0.339			& 0.661			\\
		UIB		& 0.324			& 0.676			\\
		UiT		& 0.270			& 0.730			\\
		NMBU		& 0.339			& 0.661			\\
		\hline
	\end{tabular}
	\caption{Proportion of cross institution and only same institution results}
	\label{tab:institution-proportion}
\end{figure}

\begin{figure}[h]
	\centering
	\begin{tabular}{| l || c | c |}
		\hline
		Institution	& Only same unit	& Cross unit	\\ \hline
		UiO		& 0.226			& 0.774		\\
		NTNU		& 0.543			& 0.457		\\
		UIB		& 0.505			& 0.495		\\
		UiT		& 0.320			& 0.680		\\
		NMBU		& 0.661			& 0.339		\\
		\hline
	\end{tabular}
	\caption{Proportion of cross unit and only same unit results}
	\label{tab:unit-proportion}
\end{figure}
